\chapter{Conditional Statements in Python}
\label{ch:conditional_statements}

Conditional statements let your Python programs make decisions and control the flow of execution based on whether certain conditions are \textbf{true} or \textbf{false}. They are a fundamental part of programming and real-world problem solving.

\section{Introduction to Conditional Statements}

Python provides several types of conditional statements:
\begin{itemize}
    \item \texttt{if}
    \item \texttt{if-else}
    \item \texttt{if-elif-else}
    \item Nested \texttt{if-else}
\end{itemize}

\subsection*{Note: Indentation in Python}

Indentation in Python is critical. It shows which lines of code belong to which code blocks.
\begin{itemize}
    \item \textbf{Default indentation}: 1 tab \emph{or} 4 spaces (may vary by system).
    \item \textbf{Rule}: Use consistent indentation in the same block.
\end{itemize}
\textbf{Example:}
\begin{verbatim}
if condition:
    statement1
    statement2
\end{verbatim}
Here, \texttt{statement1} and \texttt{statement2} will be executed only if the condition is true.

\section{The \texttt{if} Statement}

The simplest decision-making statement in Python is the \texttt{if} statement.

\subsection*{Syntax}
\begin{verbatim}
if <relational expression>:
    # Block of code (executed if condition is True)
    Line 1
    Line 2
    ...
\end{verbatim}

\subsection*{Example 1: Checking Zero}
\begin{verbatim}
number = input("Enter the number: ")
# The input function returns a string.
number = int(number)     # Convert to integer

if number == 0:
    print("Entered Number is ZERO")
\end{verbatim}
\textit{If the user enters 0, ``Entered Number is ZERO'' is printed.}


\section{Truthy and Falsy Values}

In Python, certain values are considered ``true'' or ``false'' in conditions (e.g., 0~$\rightarrow$~False, nonzero~$\rightarrow$~True).

\subsection*{Example 2: Truthy/Falsy Check}
\begin{verbatim}
number = input("Enter the number: ")
number = int(number)

if number:
    print("Entered Number is not ZERO")

if not number:
    print("Entered Number is ZERO")
\end{verbatim}
\textit{When \texttt{number} is 0, only the second block executes. Otherwise, the first block executes.}


\section{The \texttt{if-else} Statement}

The \texttt{if-else} statement allows one block (if) when true, another (\texttt{else}) when false.

\subsection*{Syntax}
\begin{verbatim}
if <condition>:
    # true block
else:
    # false block
\end{verbatim}

\subsection*{Example 3: Zero or Not}
\begin{verbatim}
number = input("Enter the number: ")
number = int(number)
if number == 0:
    print("Entered Number is ZERO")
else:
    print("Entered Number is not ZERO")
\end{verbatim}


\section{The \texttt{if-elif-else} Statement}

For multiple conditions, use \texttt{if-elif-else}.

\subsection*{Syntax}
\begin{verbatim}
if <condition1>:
    # block 1
elif <condition2>:
    # block 2
else:
    # block 3
\end{verbatim}

\subsection*{Example 4: Zero, One, or Others}
\begin{verbatim}
number = input("Enter the number: ")
number = int(number)

if number == 0:
    print("Entered Number is ZERO")
elif number == 1:
    print("Entered Number is ONE")
else:
    print("Others")
\end{verbatim}

\section{Practical Examples}

\subsection*{Example 5: Multiple Divisibility Checks}
\begin{verbatim}
number = input("Enter the number: ")
number = int(number)

if number % 2 == 0:
    print("Twilight")
elif number % 3 == 0:
    print("SQL")
else:
    print("PYTHON")
\end{verbatim}
\textit{Checks for divisibility by 2 (Twilight), 3 (SQL), or otherwise outputs PYTHON.}


\section{Nested \texttt{if-else} Statements}

Statements can be nested for more complex logic.

\subsection*{Syntax}
\begin{verbatim}
if <condition1>:
    if <condition2>:
        # Block A
    else:
        # Block B
else:
    # Block C
\end{verbatim}

\subsection*{Example 6: Multiple Divisibility with Nesting}
\begin{verbatim}
number = input("Enter the number: ")
number = int(number)

if number % 2 == 0:
    if number % 3 == 0:
        print("The number is divisible by 6")
    elif number % 5 == 0:
        print("The number is divisible by 10")
\end{verbatim}
\textit{Checks for divisibility by 6 or 10, only if the number is even.}


\section{Summary}

\begin{itemize}
    \item \texttt{if}: For single conditions.
    \item \texttt{if-else}: For two-way branching.
    \item \texttt{if-elif-else}: For multiple branches.
    \item \textbf{Indentation} is essential in Python for defining code blocks.
    \item Use nesting for complex decision logic.
\end{itemize}

Practice writing conditional statements to master this essential concept!

\section*{Exercises}

\begin{enumerate}
    \item Write a program that accepts marks from the user and prints ``Pass'' if marks are greater than or equal to 35, otherwise prints ``Fail''.
    \item Write a program to check if a number is positive, negative, or zero.
    \item Write a program to check if a number is odd, even, or zero.
\end{enumerate}

Conditional statements give Python programs the ability to make decisions and adapt to different situations. Master them to become a better Python programmer!
