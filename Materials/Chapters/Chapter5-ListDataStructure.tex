\chapter{List}

\section{Introduction to Lists}
A \textbf{list} in Python is a collection of data in an ordered fashion.

\begin{itemize}
	\item \textbf{Collection of data}: Lists can hold heterogeneous data; different elements of various data types can be grouped into a single list.
	\item \textbf{Ordered Fashion}: Lists are indexed; the input order and storing order will be the same and remain unaltered by the interpreter.
\end{itemize}

\subsection{Indexing in Lists}
Python follows \textbf{zero-based indexing}:
\begin{itemize}
	\item Positive indexing: From $0$ to $n-1$, where $n$ is the length of the list.
	\item Negative indexing: From $-n$ to $-1$, where $-1$ refers to the last element.
\end{itemize}

\section{Creating a List}
Lists are created using square brackets \texttt{[ ]}.

\noindent Syntax:
\begin{verbatim}
	list_variable = [value1, value2, ...]
\end{verbatim}

\noindent Example:
\begin{verbatim}
	l = [1, 2, 3, 4, "a", "e", "i", "o", "u", True]
	vowels = ["a", "e", "i", "o", "u"]
\end{verbatim}

\noindent Storage Representation:
\begin{center}
	\begin{tabular}{c|c|c|c|c|c}
		Index & 0 & 1 & 2 & 3 & 4 \\
		\hline
		Element & "a" & "e" & "i" & "o" & "u" \\
		\hline
		Negative Index & -5 & -4 & -3 & -2 & -1 \\
	\end{tabular}
\end{center}

\subsection{Index Conversion}
\begin{itemize}
	\item Negative to Positive: Add $n$ to the index.
	\item Positive to Negative: Subtract $n$ from the index.
\end{itemize}

\noindent Example:
\begin{verbatim}
	vowels[-1] == vowels[-1 + 5]
	vowels[1] == vowels[1 - 5]
\end{verbatim}

\section{Accessing List Elements}
Elements can be accessed using subscripting:
\begin{verbatim}
	vowels[0]  # First element
	vowels[-1] # Last element
\end{verbatim}

\section{Slicing of Lists}
Slicing extracts a portion of a list:
\begin{verbatim}
	list_variable[m:n]        # Elements from index m to n-1
	list_variable[m:n:step]   # With step size
\end{verbatim}

\noindent Examples:
\begin{verbatim}
	vowels = ["a", "e", "i", "o", "u"]
	vowels[1:4]      # ['e', 'i', 'o']
	vowels[-4:-1]    # ['e', 'i', 'o']
	vowels[3:]       # ['o', 'u']
	vowels[:2]       # ['a', 'e']
	vowels[::2]      # ['a', 'i', 'u']
	vowels[::-1]     # ['u', 'o', 'i', 'e', 'a']
\end{verbatim}

\noindent \textbf{Note}: Step size cannot be zero.

\section{Mutability of Lists}
Lists are mutable, meaning their elements can be altered (CRUD operations):
\begin{itemize}
	\item \textbf{C}reate
	\item \textbf{R}ead
	\item \textbf{U}pdate
	\item \textbf{D}elete
\end{itemize}

\noindent Example:
\begin{verbatim}
	vowels[0] = "p"
\end{verbatim}

\section{List Functions and Methods}

\subsection{len()}
Returns the number of elements in a list.
\begin{verbatim}
	len(vowels)
\end{verbatim}

\subsection{dir()}
Returns all attributes and methods of the object.
\begin{verbatim}
	dir(vowels)
\end{verbatim}

\subsection{sort()}
Sorts the list in ascending or descending order.
\begin{verbatim}
	vowels.sort()
	vowels.sort(reverse=True)
\end{verbatim}

\subsection{pop()}
Removes and returns the last element (or specified index).
\begin{verbatim}
	vowels.pop()
\end{verbatim}

\subsection{append()}
Adds an element to the end of the list.
\begin{verbatim}
	vowels.append("z")
\end{verbatim}

\subsection{insert()}
Inserts an element at a given index.
\begin{verbatim}
	vowels.insert(1, "b")
\end{verbatim}

\subsection{remove()}
Removes the first occurrence of the specified value.
\begin{verbatim}
	vowels.remove("a")
\end{verbatim}

\section{for Loop with Lists}
\noindent Example: Creating a list of even numbers till $n$.
\begin{verbatim}
	n = int(input("Enter n: "))
	list_even = []
	for i in range(0, n+1, 2):
	list_even.append(i)
	print(list_even)
\end{verbatim}