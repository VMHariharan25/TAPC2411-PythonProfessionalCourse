\chapter{Sets and Dictionaries}

\section{Introduction}
A \textbf{dictionary} in Python is a collection of \textbf{key-value pairs}. 
Each \textbf{key} is unique and used to access its corresponding \textbf{value}. 
Dictionaries are \textbf{mutable}, meaning they can be updated, added to, or cleared after creation.

\section{Creating a Dictionary}
Dictionaries can be created using curly braces \{\} or the \texttt{dict()} constructor.

\begin{lstlisting}[language=Python]
	# Example dictionary
	d = {"a": 1, "e": 2, "i": 3, "o": 4, "u": 5}
\end{lstlisting}

\textbf{Interactive Output:}
\begin{lstlisting}[language=Python]
	>>> d
	{'a': 1, 'e': 2, 'i': 3, 'o': 4, 'u': 5}
\end{lstlisting}

\section{Accessing Dictionary Keys}
\begin{lstlisting}[language=Python]
	>>> d.keys()
	dict_keys(['a', 'e', 'i', 'o', 'u'])
	>>> type(d.keys())
	<class 'dict_keys'>
\end{lstlisting}

Converting to a list:
\begin{lstlisting}[language=Python]
	>>> ls = list(d.keys())
	>>> ls
	['a', 'e', 'i', 'o', 'u']
\end{lstlisting}

\section{Adding New Key-Value Pairs}
\begin{lstlisting}[language=Python]
	>>> d["z"] = 9
	>>> d
	{'a': 1, 'e': 2, 'i': 3, 'o': 4, 'u': 5, 'z': 9}
\end{lstlisting}

\section{Live Views of Keys and Values}
When you store the result of \texttt{.keys()} or \texttt{.values()}, the object is a \textbf{live view} of the dictionary.  
Changes to the dictionary are automatically reflected.

\begin{lstlisting}[language=Python]
	>>> l = d.keys()
	>>> d["y"] = 10
	>>> l
	dict_keys(['a', 'e', 'i', 'o', 'u', 'z', 'y'])
\end{lstlisting}

\textbf{Tip:}
\begin{itemize}
	\item \texttt{list(d.keys())} $\rightarrow$ static copy
	\item \texttt{d.keys()} $\rightarrow$ live view
\end{itemize}

\section{Dictionary Methods}
\begin{center}
	\begin{tabular}{|l|l|l|}
		\hline
		\textbf{Method} & \textbf{Description} & \textbf{Example} \\
		\hline
		\texttt{keys()} & Returns a view of all keys & \texttt{d.keys()} \\
		\texttt{values()} & Returns a view of all values & \texttt{d.values()} \\
		\texttt{items()} & Returns (key, value) pairs & \texttt{d.items()} \\
		\texttt{get(key, default)} & Safely get a value & \texttt{d.get('a', 0)} \\
		\texttt{fromkeys(iterable, value)} & Create dict from keys with same value & \texttt{dict.fromkeys(['a','b'], 0)} \\
		\texttt{update(other\_dict)} & Add/overwrite from another dict & \texttt{d.update(\{'x': 6\})} \\
		\texttt{pop(key)} & Remove and return value for key & \texttt{d.pop('a')} \\
		\texttt{popitem()} & Remove and return last inserted pair & \texttt{d.popitem()} \\
		\texttt{clear()} & Remove all items & \texttt{d.clear()} \\
		\hline
	\end{tabular}
\end{center}

\section{Looping Through Dictionaries}
\begin{lstlisting}[language=Python]
	# Keys only
	for key in d:
	print(key)
	
	# Keys and values
	for key, value in d.items():
	print(key, value)
\end{lstlisting}

\section{Advanced Dictionary Operations}

\subsection{Dictionary Comprehensions}
\begin{lstlisting}[language=Python]
	# Squares of numbers from 1 to 5
	squares = {x: x**2 for x in range(1, 6)}
	print(squares)
	# Output: {1: 1, 2: 4, 3: 9, 4: 16, 5: 25}
\end{lstlisting}

\subsection{Conditional Dictionary Comprehensions}
\begin{lstlisting}[language=Python]
	# Only even squares
	even_squares = {x: x**2 for x in range(1, 6) if x % 2 == 0}
	# Output: {2: 4, 4: 16}
\end{lstlisting}

\subsection{Merging Dictionaries}
Python 3.9+:
\begin{lstlisting}[language=Python]
	d1 = {"a": 1, "b": 2}
	d2 = {"b": 3, "c": 4}
	merged = d1 | d2
	# Output: {'a': 1, 'b': 3, 'c': 4}
\end{lstlisting}

Before Python 3.9:
\begin{lstlisting}[language=Python]
	merged = {**d1, **d2}
\end{lstlisting}

\subsection{Nested Dictionaries}
\begin{lstlisting}[language=Python]
	students = {
		"Alice": {"math": 90, "science": 85},
		"Bob": {"math": 78, "science": 82}
	}
	print(students["Alice"]["math"])  # Output: 90
\end{lstlisting}

\subsection{Inverting Dictionaries}
\begin{lstlisting}[language=Python]
	d = {"a": 1, "b": 2, "c": 3}
	inv = {v: k for k, v in d.items()}
	print(inv)
	# Output: {1: 'a', 2: 'b', 3: 'c'}
\end{lstlisting}

\subsection{Using \texttt{defaultdict} for Automatic Values}
\begin{lstlisting}[language=Python]
	from collections import defaultdict
	dd = defaultdict(int)
	dd["x"] += 1
	print(dd)  
	# Output: defaultdict(<class 'int'>, {'x': 1})
\end{lstlisting}

\subsection{Sorting Dictionaries}
\begin{lstlisting}[language=Python]
	# Sort by keys
	sorted_by_keys = dict(sorted(d.items()))
	
	# Sort by values
	sorted_by_values = dict(sorted(d.items(), key=lambda item: item[1]))
\end{lstlisting}

\section{Summary}
\begin{itemize}
	\item Dictionaries store unique keys with associated values.
	\item \texttt{.keys()}, \texttt{.values()}, and \texttt{.items()} return live views.
	\item Keys must be immutable, values can be any type.
	\item Advanced features include comprehensions, merging, nested structures, and sorting.
\end{itemize}